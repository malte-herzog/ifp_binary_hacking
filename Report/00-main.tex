\documentclass[conference,dvipsnames]{IEEEtran}
\IEEEoverridecommandlockouts
% The preceding line is only needed to identify funding in the first footnote. If that is unneeded, please comment it out.
\usepackage{cite}
\usepackage{amsmath,amssymb,amsfonts}
\usepackage{algpseudocode}
\usepackage{graphicx}
\usepackage{float} % enables [H] float placement
\ifCLASSOPTIONcompsoc
  \usepackage[caption=false,font=normalsize,labelfont=sf,textfont=sf]{subfig}
\else
  \usepackage[caption=false,font=footnotesize]{subfig}
\fi
\usepackage{textcomp}
\usepackage{xcolor}
\usepackage{todonotes}
\usepackage{booktabs}
\usepackage{url}
\usepackage{tikz}
\usetikzlibrary{trees, matrix, positioning, patterns, shapes, shadows, shapes.arrows, arrows.meta, shapes.multipart, decorations.pathreplacing, calc, tikzmark, shapes.geometric}
\usepackage{pgfplots}
\pgfplotsset{compat=1.17}
\usepackage{pgf-pie}
\usepackage[binary-units]{siunitx}
\usepackage{mathdots}
\usepackage{mathtools}
\usepackage{bm}
\usepackage[colorlinks=true, linkcolor=black, urlcolor=NavyBlue, citecolor=black]{hyperref}

\newcommand{\MatVal}[1]{\bm{\mathit{#1}}}

\def\BibTeX{{\rm B\kern-.05em{\sc i\kern-.025em b}\kern-.08em
    T\kern-.1667em\lower.7ex\hbox{E}\kern-.125emX}}
\begin{document}

\title{Binary Hacking:\\Reverse Engineering von "Mole Mania"}

\author{
\IEEEauthorblockN{Malte Herzog, Luke Böhler, Cemile Büyükkakac}
\IEEEauthorblockA{\textit{Fortgeschrittenen Praktikum, Informatik - Bachelor} \\
\textit{Universität Heidelberg, Deutschland}\\
malte.herzog.klemenz@gmail.com}
}

\maketitle

\begin{abstract}
	Unser Ziel war es, das Gameboy Spiel "Mole Mania" zu zerlegen und ROM-Hacking zu betreiben.
  Dafür nutzten wir verschiedene Reverse Engineering Tools, die uns im Rahmen des Praktikums 
  zur Verfügung standen, insbesondere Radare2. Insbesondere veränderten wir logische und 
  graphische Aspekte des Spieles.
\end{abstract}

% \begin{IEEEkeywords}
% component, formatting, style, styling, insert
% \end{IEEEkeywords}

\section{Mole Mania Basics - Die generelle Spiellogik}
\subsection{Einleitung}
In diesem klassischen Action-Knobelspiel, welches vom berühmten Shigeru Miyamoto entwickelt wurde,
spielt man einen Maulwurf namens "Muddy Mole" der seine Familie vor dem bösen Bauern Jinbe retten will. \\
Für das weitere Verständis unseres Reverse Engineering ist es wichtig, ein grundlegendes, oberflächliches 
Verständnis vom Spiel selbst zu erlangen. \\
Generell ist es deine Aufgabe, Muddy zum Ausgang des Levels zu führen und am Ende einen Boss zu besiegen. \\
Zur Veranschaulichung zeigen die nächsten zwei Bilder zwei Räume, die man durchqueren muss, um das erste 
Level zu schaffen. \\

\begin{figure}[H]
    \centering
    % Erstes Bild
    \begin{minipage}{0.45\linewidth}
        \centering
        \includegraphics[width=\linewidth]{Bilder/Bild1.png}
        \caption{irgendwo im ersten Level}
        \label{fig:enter-label}
    \end{minipage}
    \hfill
    % Zweites Bild
    \begin{minipage}{0.45\linewidth}
        \centering
        \includegraphics[width=\linewidth]{Bilder/Bild2.png}
        \caption{irgendwo im ersten Level}
        \label{fig:enter-label}
    \end{minipage}
\end{figure}

Die grün markierten Felder sind die Startfelder, wo man anfängt wenn man den Raum betritt. Das rote Feld 
ist das Ziel, dort kommt man in den nächsten Raum. \\
Alle blau markierten Objekte sind Objekte mit dem man auf irgendeiner Art und Weise interagiert. Ein paar 
Objekte helfen einen um ans Ziel zu kommen und ein paar Objekte hindern einen um ans Ziel zu kommen. \\

\subsection{Muddy Mole}
Wie bereits schon erwähnt spielt man einen Maulwurf, der mit allem möglichem in einem Raum interagieren kann. 
An seine Grenzen kommt er, wenn er 4 mal Schaden erlitten hat, dann folgt daraufhin ein Raum Neustart.\\
Was dem Spiel einen ganz eigenen Charakter gibt, ist die Möglichkeit sich in die Erde zu graben. Dadurch
hat quasi jeder Raum eine zwei Spielflächen: Eine Oberfläche, oft mit Gegnern und einen unterirdischen Bereich,
wo oft Items zu finden sind. Einschränkungen bilden sich dadurch wieder, dass man sich nicht überall runtergraben 
kann und nicht überall hingraben kann.\\

\begin{figure}[H]
    \centering
    % Erstes Bild
    \begin{minipage}{0.45\linewidth}
        \centering
        \includegraphics[width=\linewidth]{Bilder/Bild3.png}
        \caption{obere Spielfläche}
        \label{fig:enter-label}
    \end{minipage}
    \hfill
    % Zweites Bild
    \begin{minipage}{0.45\linewidth}
        \centering
        \includegraphics[width=\linewidth]{Bilder/Bild4.png}
        \caption{unterirdische Spielfläche}
        \label{fig:enter-label}
    \end{minipage}
\end{figure}
\clearpage
\section{Der Game Boy und unsere Cardridge}

Mole Mania ist für den Game Boy (Game Boy Classic) entwickelt worden. 
Der Game Boy hat eine Sharp LR35902 CPU verbaut, welches eine Mischung aus 
dem Intel 8080 und dem Zilog Z80 ist. Das heißt, es handelt sich um eine 
8 Bit CPU mit einem 16 bit Adressraum. 

\begin{figure}[H]
    \centering
    \includegraphics[width=0.5\linewidth]{Bilder/Gameboy.jpg}
    \caption{Nintendo Game Boy Classic}
    \label{fig:enter-label}
\end{figure}

Der Speicheraufbau des Game Boy's sieht folgendermaßen aus:
\begin{itemize}
    \item 0000 - 3FFF: ROM Bank00
    \item 4000 - 7FFF: ROM Bank01 - 07
    \item 8000 - 9FFF: VRAM
    \item A000 - BFFF: External RAM
    \item C000 - DFFF: WRAM
    \item E000 - FFFF: Echo RAM, OAM, I/O Register, HRAM \\
\end{itemize}

Wir nutzten Emulatoren, die das Spiel auf einem Nicht-Game-Boy-System ausführen 
konnten, sodass es wirkte, als liefe es in seiner ursprünglichen Umgebung. Besonders 
wichtig für unser Projekt war dabei die Nachbildung von Prozessor und Speicher.
\section{Unsere Tools}

Verschiedene Tools, die uns vom Praktikum oder die wir selber gefunden haben, halfen uns beim verstehen des 
Speilflusses und beim schreiben unserer Cheats.

\subsection{Unsere Emulatoren}
Als Emulatoren benutzten wir hauptsächlich mGBA und SameBoy. Die genannten Emulatoren haben weitere Funktionen 
die uns geholfen haben mit dem reproduziertem System zu arbeiten:
\begin{itemize}
    \item Memory-Search
    \item RAM-Viewer
    \item VRAM-Viewer
    \item Debugger
\end{itemize}

\subsection{Unsere Reverse Engineering Tools}

Um unsere .gb Datei zu disassemblen, benutzten wir ausschließlich Radare2 und Ghidra. \\

Ghidra hatte die Funktionaliät, den Assemblercode in Pseudo-C-Code darzustellen. Es stellte sich jedoch heraus, 
dass die fehlenden Variablen- und Funktionsnamen ein wesentlicher Aspekt waren, weshalb der C-Code für uns nicht wesentlich leichter zu verstehen war. \\

Da uns Radare2 im Praktikum vorgestellt worden ist und wir eine Palette an Funktionen erklärt bekommen haben, konnten wir einiges damit herausfinden.\\
Häufig verwendete Befehle: 
\begin{itemize}
    \item $/$x (zum suchen exakter Hexzahlenfolgen)
    \item axt (findet Referenz zu dieser Adresse)
\end{itemize}

\subsection{Python}
Für unseren Wrapper, um unsere Cheats spielerisch zu gestalten, benutzten wir Python als Programmiersprache. 
Benutzte Libraries:
\begin{itemize}
    \item tkinter
\end{itemize}

\clearpage
\section{Your First Question Here}

\subsection{Problem}

Describe the problem the interviewee is expected to solve. If applicable
mention constraints like runtime or memory complexity, coding style, edge
cases, etc. If you use information from external sources,
like~\cite{McDowell_Cracking_2015}, consider a citation if it is a book or
article. For blog entries or other links a footnote with a URL will suffice.

\subsection{Baseline Solution}

Give the solution to the previously described problem. Talk about possible
pitfalls and hints you could give the interviewee. Show code snippets where
applicable. You may, of course, also use graphics and plots.

\subsection{(Optional) Post-Interview Experiences}

How was the question received, did interviewees struggle with a particular part
the most.

\input{05-question02}
\input{06-question03}

% Only if you have references
\bibliographystyle{IEEEtran}
\bibliography{bibliography}

\clearpage
%\input{appendix.tex}


\end{document}
