\section{Der Game Boy und unsere Cardridge}

Mole Mania ist für den Game Boy (Game Boy Classic) entwickelt worden. 
Der Game Boy hat eine Sharp LR35902 CPU verbaut, welches eine Mischung aus 
dem Intel 8080 und dem Zilog Z80 ist. Das heißt, es handelt sich um eine 
8 Bit CPU mit einem 16 bit Adressraum. 

\begin{figure}[H]
    \centering
    \includegraphics[width=0.5\linewidth]{Bilder/Gameboy.jpg}
    \caption{Nintendo Game Boy Classic}
    \label{fig:enter-label}
\end{figure}

Der Speicheraufbau des Game Boy's sieht folgendermaßen aus:
\begin{itemize}
    \item 0000 - 3FFF: ROM Bank00
    \item 4000 - 7FFF: ROM Bank01 - 07
    \item 8000 - 9FFF: VRAM
    \item A000 - BFFF: External RAM
    \item C000 - DFFF: WRAM
    \item E000 - FFFF: Echo RAM, OAM, I/O Register, HRAM \\
\end{itemize}

Wir nutzten Emulatoren, die das Spiel auf einem Nicht-Game-Boy-System ausführen 
konnten, sodass es wirkte, als liefe es in seiner ursprünglichen Umgebung. Besonders 
wichtig für unser Projekt war dabei die Nachbildung von Prozessor und Speicher.