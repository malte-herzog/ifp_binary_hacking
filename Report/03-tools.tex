\section{Unsere Tools}

Verschiedene Tools, die uns vom Praktikum oder die wir selber gefunden haben, halfen uns beim verstehen des 
Speilflusses und beim schreiben unserer Cheats.

\subsection{Unsere Emulatoren}
Als Emulatoren benutzten wir hauptsächlich mGBA und SameBoy. Die genannten Emulatoren haben weitere Funktionen 
die uns geholfen haben mit dem reproduziertem System zu arbeiten:
\begin{itemize}
    \item Memory-Search
    \item RAM-Viewer
    \item VRAM-Viewer
    \item Debugger
\end{itemize}

\subsection{Unsere Reverse Engineering Tools}

Um unsere .gb Datei zu disassemblen, benutzten wir ausschließlich Radare2 und Ghidra. \\

Ghidra hatte die Funktionaliät, den Assemblercode in Pseudo-C-Code darzustellen. Es stellte sich jedoch heraus, 
dass die fehlenden Variablen- und Funktionsnamen ein wesentlicher Aspekt waren, weshalb der C-Code für uns nicht wesentlich leichter zu verstehen war. \\

Da uns Radare2 im Praktikum vorgestellt worden ist und wir eine Palette an Funktionen erklärt bekommen haben, konnten wir einiges damit herausfinden.\\
Häufig verwendete Befehle: 
\begin{itemize}
    \item $/$x (zum suchen exakter Hexzahlenfolgen)
    \item axt (findet Referenz zu dieser Adresse)
\end{itemize}

\subsection{Python}
Für unseren Wrapper, um unsere Cheats spielerisch zu gestalten, benutzten wir Python als Programmiersprache. 
Benutzte Libraries:
\begin{itemize}
    \item tkinter
\end{itemize}
