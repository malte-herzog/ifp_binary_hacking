\section{Mole Mania Basics - Die generelle Spiellogik}
\subsection{Einleitung}
In diesem klassischen Action-Knobelspiel, welches vom berühmten Shigeru Miyamoto entwickelt wurde,
spielt man einen Maulwurf namens "Muddy Mole" der seine Familie vor dem bösen Bauern Jinbe retten will. \\
Für das weitere Verständis unseres Reverse Engineering ist es wichtig, ein grundlegendes, oberflächliches 
Verständnis vom Spiel selbst zu erlangen. \\
Generell ist es deine Aufgabe, Muddy zum Ausgang des Levels zu führen und am Ende einen Boss zu besiegen. \\
Zur Veranschaulichung zeigen die nächsten zwei Bilder zwei Räume, die man durchqueren muss, um das erste 
Level zu schaffen. \\

\begin{figure}[H]
    \centering
    % Erstes Bild
    \begin{minipage}{0.45\linewidth}
        \centering
        \includegraphics[width=\linewidth]{Bilder/Bild1.png}
        \caption{irgendwo im ersten Level}
        \label{fig:enter-label}
    \end{minipage}
    \hfill
    % Zweites Bild
    \begin{minipage}{0.45\linewidth}
        \centering
        \includegraphics[width=\linewidth]{Bilder/Bild2.png}
        \caption{irgendwo im ersten Level}
        \label{fig:enter-label}
    \end{minipage}
\end{figure}

Die grün markierten Felder sind die Startfelder, wo man anfängt wenn man den Raum betritt. Das rote Feld 
ist das Ziel, dort kommt man in den nächsten Raum. \\
Alle blau markierten Objekte sind Objekte mit dem man auf irgendeiner Art und Weise interagiert. Ein paar 
Objekte helfen einen um ans Ziel zu kommen und ein paar Objekte hindern einen um ans Ziel zu kommen. \\

\subsection{Muddy Mole}
Wie bereits schon erwähnt spielt man einen Maulwurf, der mit allem möglichem in einem Raum interagieren kann. 
An seine Grenzen kommt er, wenn er 4 mal Schaden erlitten hat, dann folgt daraufhin ein Raum Neustart.\\
Was dem Spiel einen ganz eigenen Charakter gibt, ist die Möglichkeit sich in die Erde zu graben. Dadurch
hat quasi jeder Raum eine zwei Spielflächen: Eine Oberfläche, oft mit Gegnern und einen unterirdischen Bereich,
wo oft Items zu finden sind. Einschränkungen bilden sich dadurch wieder, dass man sich nicht überall runtergraben 
kann und nicht überall hingraben kann.\\

\begin{figure}[H]
    \centering
    % Erstes Bild
    \begin{minipage}{0.45\linewidth}
        \centering
        \includegraphics[width=\linewidth]{Bilder/Bild3.png}
        \caption{obere Spielfläche}
        \label{fig:enter-label}
    \end{minipage}
    \hfill
    % Zweites Bild
    \begin{minipage}{0.45\linewidth}
        \centering
        \includegraphics[width=\linewidth]{Bilder/Bild4.png}
        \caption{unterirdische Spielfläche}
        \label{fig:enter-label}
    \end{minipage}
\end{figure}